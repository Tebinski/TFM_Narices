%% ARROW
\newcommand{\arrowTikz}[1]
{
	\begin{tikzpicture}[rotate=#1]
		\coordinate (initPoint)   at (0,0);
		\coordinate (endingPoint) at (0.5,0);
		\draw [line width=1pt,-{Stealth[length=3pt,width=4pt,inset=0.3pt]}](initPoint)--(endingPoint);
	\end{tikzpicture}
}

%% SECTION AND SUBSECTIONS REFERENCE
\newcommand{\refsec}[1]{section~\ref{#1}}
\newcommand{\refsubsec}[1]{subsecsection~\ref{#1}}

%% SUPER & SUB SCRIPT
\newcommand{\superscript}[1]{\ensuremath{^{\textrm{#1}}}}
\newcommand{\subscript}[1]{\ensuremath{_{\textrm{#1}}}}
%% BOLD ITEM
\newcommand\litem[1]{\item{\bfseries #1\enspace} \\}
%% WRITE A VECTOR IN BOLD
\newcommand{\bvec}[1]{\vec{\mathbf{#1}}}
%% PARTIAL DERIVATIVE
\newcommand{\pdv}[2]{\frac{\partial #1}{\partial #2}}
%% BIGO
\DeclareMathAlphabet{\mathpzc}{OT1}{pzc}{m}{it}
\newcommand{\bigO}[1]{$\mathpzc{O}(#1)$}
%% FOR TABLES
\newfloatcommand{capbtabbox}{table}[][\FBwidth]
%% EARTH SPHERE DRAWING
\newcommand\pgfmathsinandcos[3]{%
	\pgfmathsetmacro#1{sin(#3)}%
	\pgfmathsetmacro#2{cos(#3)}%
}
\newcommand\LongitudePlane[3][current plane]{%
	\pgfmathsinandcos\sinEl\cosEl{#2} % elevation
	\pgfmathsinandcos\sint\cost{#3} % azimuth
	\tikzset{#1/.style={cm={\cost,\sint*\sinEl,0,\cosEl,(0,0)}}}
}
\newcommand\LatitudePlane[3][current plane]{%
	\pgfmathsinandcos\sinEl\cosEl{#2} % elevation
	\pgfmathsinandcos\sint\cost{#3} % latitude
	\pgfmathsetmacro\yshift{\cosEl*\sint}
	\tikzset{#1/.style={cm={\cost,0,0,\cost*\sinEl,(0,\yshift)}}} %
}
\newcommand\DrawLongitudeCircle[2][1]{
	\LongitudePlane{\angEl}{#2}
	\tikzset{current plane/.prefix style={scale=#1}}
	% angle of "visibility"
	\pgfmathsetmacro\angVis{atan(sin(#2)*cos(\angEl)/sin(\angEl))} %
	\draw[current plane] (\angVis:1) arc (\angVis:\angVis+180:1);
	\draw[current plane,dashed] (\angVis-180:1) arc (\angVis-180:\angVis:1);
}
\newcommand\DrawLongitudeCircleRed[2][1]{
	\LongitudePlane{\angEl}{#2}
	\tikzset{current plane/.prefix style={scale=#1}}
	% angle of "visibility"
	\pgfmathsetmacro\angVis{atan(sin(#2)*cos(\angEl)/sin(\angEl))} %
	\draw[current plane, color = red] (\angVis:1) arc (\angVis:\angVis+180:1);
	\draw[current plane,dashed, color = red] (\angVis-180:1) arc (\angVis-180:\angVis:1);
}
\newcommand\DrawLatitudeCircle[2][2]{
	\LatitudePlane{\angEl}{#2}
	\tikzset{current plane/.prefix style={scale=#1}}
	\pgfmathsetmacro\sinVis{sin(#2)/cos(#2)*sin(\angEl)/cos(\angEl)}
	% angle of "visibility"
	\pgfmathsetmacro\angVis{asin(min(1,max(\sinVis,-1)))}
	\draw[current plane] (\angVis:1) arc (\angVis:-\angVis-180:1);
	\draw[current plane,dashed] (180-\angVis:1) arc (180-\angVis:\angVis:1);
}
\newcommand\DrawLatitudeCircleRed[2][2]{
	\LatitudePlane{\angEl}{#2}
	\tikzset{current plane/.prefix style={scale=#1}}
	\pgfmathsetmacro\sinVis{sin(#2)/cos(#2)*sin(\angEl)/cos(\angEl)}
	% angle of "visibility"
	\pgfmathsetmacro\angVis{asin(min(1,max(\sinVis,-1)))}
	\draw[current plane,red] (\angVis:1) arc (\angVis:-\angVis-180:1);
	\draw[current plane,dashed,red] (180-\angVis:1) arc (180-\angVis:\angVis:1);
}

%% CAPTION FOR EQUATION SET
\newcounter{equationset}
\newcommand{\equationset}[1]{% \equationset{<caption>}
	\refstepcounter{equationset}% Step counter
	\noindent\makebox[\linewidth]{Ecuaci\'on~\theequationset: #1}}% Print caption