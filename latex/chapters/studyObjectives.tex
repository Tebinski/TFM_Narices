\chapter{Objetivos del trabajo}

Dados unos datos experimentales provenientes de una nariz electrónica,
\textbf{insertar aquí enlace UCI data} se desea hacer un estudio de la viabilidad de
clasificar correctamente de qué gas se trata,
e intentar dar una estimación de la concentración de dicho gas.

Los datos provienen de cada medición de un gas con 16 sensores.
De esta forma, cada medición genera 16 series temporales, de las cuales de cada una se
han extraido 8 \emph{features}. Esto hace un total de 128 componentes para cada medición.
Para cada medición es conocido el gas que se ha ensayado y su nivel de concentración.

Con este tipo de información, se va a estudiar los resultados que pueden ofrecer las redes neuronales
para resolver este problema.

En el caso que nos ocupa, los sensores derivan a
lo largo del tiempo, lo cual implica que para el mismo gas, con la misma concentración, las series temporales obtenidas
en un mes dado podrían ser diferentes a las realizadas con las mismas condiciones meses después.

En este trabajo se tomará como primera aproximación dividir la tarea de clasificación y la tarea de estimación.

Para la tarea de clasificación, se probarán las siguientes configuraciones, de más simple a más compleja:

\begin{itemize}
    \item Perceptron
    \item \textbf{Long short-term memory \glossary{LSTM} Neural net}
\end{itemize}

Una vez conseguida la tarea de clasificación, se usará otra red neuronal para determinar, si es posible,
la concentración del gas.

En este trabajo se utilizará Keras y TensorFlow para el diseño de las redes neuronales, en el entorno Python.

